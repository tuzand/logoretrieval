% ~~~~~~~~~~~~~~~~~~~~~~~~~~~~~~~~~~~~~~~~~~~~~~~~~~~~~~~~~~~~~~~~~~~~~~~~
% Fonts Fonts Fonts
% ~~~~~~~~~~~~~~~~~~~~~~~~~~~~~~~~~~~~~~~~~~~~~~~~~~~~~~~~~~~~~~~~~~~~~~~~

% Alle Schriften die hier angegeben sind sehen im PDF richtig aus.
% Die LaTeX Standardschrift ist die Latin Modern (lmodern Paket).
% If Latin Modern is not available for your distribution you must install the
% package cm-super instead. Otherwise your fonts will look horrible in the PDF

% DO NOT LOAD ae Package for the font !



%% PW: NFSS-Erweiterungen wie zB condensed. Eigentlich wegen venturisadf drin.
\usepackage{nfssext-cfr}

%% PW: Initialen
\usepackage{lettrine}

%% ==== Zusammengesetzte Schriften  (Sans + Serif) =======================

%% - Latin Modern
%\usepackage{lmodern}  %%% PW: muss trotz mathpazo o.�. rein, damit nicht die ec-Schriften im pk-Format geladen werden. Die sind nicht skalierbar und verhindern, dass microtype anst�ndig l�uft.  %%Nochmal probiert und gab irgendwie doch keine Probleme mehr mit microtype

%\usepackage{kpfonts}

%%%%%%%%%%%% KPfonts mit Euler Math: Microtype: Letterspace -5
%\usepackage[sc,slantedGreek]{mathpazo}   %% PW mit smallcaps-Option wird ein anderer Font mit besserem Kerning verwendet  %%%% Nach Mathpazo ist mathrm in Palatino statt in kpfonts.
%\usepackage[mathpazo]{flexisym}  %%PW: Geh�rt zu breqn (s.u.)
%\usepackage{pxfonts} %%PW: Kompletterer Satz an Zeichen als Math Pazo

%\usepackage[%
%%light,%
%%fulloldstylenums,%
%%fulloldstyle,%
%%fullveryoldstyle,%
%nomath,% killt auch und insb. die mathbb-Variante. Sorgt aber daf�r, dass die Oldstyle-Zahlen in Euler-Mathfrak funktionieren
%%notext,%
%%nosf,%
%%nott,%
%%onlyrm,%
%%noamsmath,%
%%notextcomp,%
%%lighttext,%
%%oldstylenums,%
%%oldstyle,%
%%veryoldstyle,%
%%rmx,%
%%easyscsl,%
%%nofligatures,%
%%%%Math
%%lighttext,%
%%sfmath,%
%%sfmathbb,%
%%rmmathbb,%
%%nomathscript,%
%%mathcalasscript,%
%%classicReIm,%
%%uprightRoman,%
%%frenchstyle,%
%%oldstylenumsmath,%
%%oldstylemath,%
%%veryoldstylemath,%
%%narrowiints,%
%%partialup,% macht im Euler-Math aus partial ein hslash
%intlimits,%
%%widermath,%
%%noDcommand,%
%largesmallcaps%
%]{kpfonts}

\usepackage{libertine}  %% F�r Biolinum, Freier Optima-Klon
%\biolinum

\usepackage{fourier}  %%PW: Nach den kpfonts laden, sonst textcomp-Fehler. M�glicherweise w�rde es auch gehen, wenn man bei den kpfonts die Option notextcomp verwendet.
%\usepackage[scaled=0.85]{berasans}

%\cdwidth ist condensed bzw. \textcd

%% PW Die beiden ersten geh�ren zu venturis. Brachten aber nix, hat trotzdem nicht geklappt
%\usepackage{xkeyval}
%\usepackage{nfssext-cfr}  %% s.o. ist schon eingebunden

%\usepackage[%
%lf%  %%lining figures im Gegensatz zu osf (default) old style figures
%]{venturis}  %%Erzeugt keinen Fehler obwohl das Paket eigentlich venturisadf hei�t. Siehe venturisadf Dokumentation.

%% Hat nicht geklappt weil noch Problem mit MikTeX-Package, Aktivierung von Hand:
%0. Installation von Paket venturisadf �ber den MikTeX-Paketmanager. Automatisch gings irgendwie nicht :-(
%1. initexmf --edit-config-file updmap		%Da folgendes reinschreiben
%
%# Maps for using PS version of VenturisADF fonts
%Map yvt.map
%Map yv2.map
%Map yv1.map
%Map yv3.map
%Map yvo.map
%# end maps for VenturisADF
%
%2. initexmf --mkmaps
%3. initexmf -u


%\usepackage{pandora}  %%PW: Bitmapschrift! Kollidiert mit Microtype

%% PW: Optima clone: usepackage{urw-classico} oder usepackage{urw-classico}. Ist aber nicht direkt in LaTeX Distri wegen Lizenz

\renewcommand{\ttdefault}{lmtt}
%\renewcommand{\sfdefault}{lmss}        %%PW --- Latin Modern Sans f�r �berschriften
\renewcommand{\sfdefault}{fxb}  %fxb ist Biolinum. Keine Kursive!
%\renewcommand{\sfdefault}{yv1}  %yv1 ist venturis sans
%\renewcommand{\sfdefault}{yv3}  %yv3 ist venturis2 sans
\setkomafont{sectioning}{\normalcolor\bfseries}

%%PW: Palatino definiert mathrm und mathbf um, aber kpfonts �berschreibt es nicht.
%% Mathrm und Mathbf sollen wieder aus den kpfonts kommen, Mathsf ist wenn man nichts macht aus lmodern, wird auch auf kpfonts umdefiniert.
%\DeclareMathAlphabet{\mathrm} {OT1}{jkp}{m}{n}
%\DeclareMathAlphabet{\mathbf} {OT1}{jkp}{b}{n}
%\DeclareMathAlphabet{\mathsf} {OT1}{jkpss}{m}{n}


%\DeclareSymbolFont{AMSa}{U}{msa}{m}{n}
%\DeclareSymbolFont{AMSb}{U}{msb}{m}{n}

%\DeclareFontFamily{U}{msb}{}
%\DeclareFontShape{U}{msb}{m}{n}{ <5> <6> <7> <8> <9> gen * msbm <10> <10.95> <12> <14.4> <17.28> <20.74> <24.88> msbm10}  %aus LaTeX-Begleiter S.445. Gibt Problem: Er kann die msbm10 nicht finden, kein TFM file dazu.
%\DeclareFontShape{U}{msb}{m}{n}{msbm10}
%\DeclareSymbolFont{AMSb}{U}{msb}{m}{n}
%\SetSymbolFont{AMSb}{bold}{U}{msb}{b}{n}

%\DeclareSymbolFontAlphabet{\mathbb}{AMSb}

\DeclareSymbolFont{amscal}{OMS}{cmsy}{m}{n}
\SetSymbolFont{amscal}{bold}{OMS}{cmsy}{b}{n}

\DeclareSymbolFont{eulercal}{U}{zeus}{m}{n}
\SetSymbolFont{eulercal}{bold}{U}{zeus}{b}{n}


%%%PW: Mathcal und MathBB wieder r�ckdefinieren
\AtBeginDocument{%
  \let\mathbb\relax
	%%\DeclareMathAlphabet\amsa{U}{msa}{m}{n}
		\DeclareSymbolFont{AMSb}{U}{msb}{m}{n}
	%	\SetSymbolFont{AMSb}{bold}{U}{msb}{b}{n} %gibt kein bold bei der msb laut LaTeX-Begleiter. S.446
	%\DeclareMathAlphabet{\amsb}{U}{msb}{m}{n}  % habe das Gef�hl: Declare...Alphabet f�hrt viel fr�her zum "`too many alphabets"' Fehler
	%besser DeclareSymbolFontAlphabet laut LaTeX-Begleiter, wenn die Symbole bereits geladen sind (S.448)
	\DeclareSymbolFontAlphabet{\mathbb}{AMSb}
	%\SetMathAlphabet{\mathbb}{normal}{AMSb}
	%\SetMathAlphabet{\mathbb}{normal}{U}{msb}{m}{n}
	%\newcommand{\mathbb}{\amsb}
}
	
\AtBeginDocument{%
  %\let\mathcal\relax
	%\DeclareMathAlphabet\amscal{OMS}{cmsy}{m}{n}
	  %\SetMathAlphabet    \mathcal{normal}{OMS}{cmsy}{m}{n}
  %\DeclareMathAlphabet\mathbcal{OMS}       {cmsy}{b}{n}
	%besser DeclareSymbolFontAlphabet laut LaTeX-Begleiter, wenn die Symbole bereits geladen sind (S.448)
	\DeclareSymbolFontAlphabet{\mathcal}{amscal}  %braucht keines der 16 Register
  %\newcommand{\mathcal}{\amscal}
}
%
%
%%%%Mathscr definieren. Braucht man gar nicht, weil eucal daf�r da ist
\AtBeginDocument{%
  %\let\mathscr\relax
	%\DeclareMathAlphabet\eulercal{U}{zeus}{m}{n}
  %\DeclareMathAlphabet\mathbscr{U}       {zeus}{b}{n}
	  %\SetMathAlphabet    \mathscr{normal}{U}{zeus}{m}{n}
	\DeclareSymbolFontAlphabet{\mathscr}{eulercal}
  %\newcommand{\mathcal}{\eulercal}
}

%%Aus mathpazo
%\AtBeginDocument{%
  %\let\mathbb\relax
  %\DeclareMathAlphabet\PazoBB{U}{fplmbb}{m}{n}
  %\newcommand{\mathbb}{\PazoBB}
%}



%%%%PW: Mathcal wieder r�ckdefinieren	
%\AtBeginDocument{%
  %\let\mathcal\relax
	%\DeclareMathAlphabet\amscal{OMS}{cmsy}{m}{n}
	  %\SetMathAlphabet    \mathcal{normal}{OMS}{cmsy}{m}{n}
	%\DeclareMathAlphabet\mathbcal{OMS}       {cmsy}{b}{n}
	%%besser DeclareSymbolFontAlphabet laut LaTeX-Begleiter, wenn die Symbole bereits geladen sind (S.448)
	%%\DeclareSymbolFontAlphabet{\mathcal}{amscal}  %braucht keines der 16 Register
  %\newcommand{\mathcal}{\amscal}
%}
%%

%%%PW: Mathfrak wieder r�ckdefinieren	
%\AtBeginDocument{%
  %\let\mathcal\relax
	%\DeclareMathAlphabet\amscal{OMS}{cmsy}{m}{n}
	  %\SetMathAlphabet    \mathcal{normal}{OMS}{cmsy}{m}{n}
	%\DeclareMathAlphabet\mathbcal{OMS}       {cmsy}{b}{n}
	%%besser DeclareSymbolFontAlphabet laut LaTeX-Begleiter, wenn die Symbole bereits geladen sind (S.448)
	%%\DeclareSymbolFontAlphabet{\mathcal}{amscal}  %braucht keines der 16 Register
  %\newcommand{\mathcal}{\amscal}
%}
%%%



%%%% PW: Breites Dach-Symbol definieren. Wichtig f�r einige newcommand-Definitionen sp�ter
\DeclareMathSymbol{\widehatsym}{\mathord}{largesymbols}{"62}





% Recommanded to use with fonts: Aldus, Garamond, Melior, Sabon

%%PW: Achtung: Wenn auskommentiert kein Mathbold-Befehl!!
%\usepackage[                           %% --- EulerVM (MATH)
%   small,       %for smaller Fonts, 95% Normalgr��e
%   %euler-hat-accent  %Euler dach-Akzent
%   euler-digits % digits in euler fonts style
%]{eulervm}

%%PW: Euler Script: Gibt dann aber Fehler wegen zu vielen Mathealphabeten
%\usepackage[%
%mathscr%
%]%
%{eucal}%

\usepackage{mathdots}  %%PW: fixt Problem mit den \dots Befehlen
%\usepackage{eufrak}  %% PW: Euler Fraktur   %%knallt mit amssymb, was das gleiche macht

%%PW alternative Blackboard Mengenbuchstaben. Brauchen auch Math alphabet
%\usepackage[%
%%sans%
%]{dsfont} 

%% PW
\renewcommand*{\marginfont}{%
\scriptsize
%\color{red}
\sffamily
}
%%%%%

% PW:
\usepackage{pifont} % numbers in circles

%%PW:
%Palatino text with Euler math
%\usepackage{mathpple}


%%%%%%%%%%%%% Utopia with Fourier Math
%\usepackage{fourier}  %%Doch, geht mit Microtype. Fehler beim Unendlich-Zeichen (Ist Musik-Bindungsbogen) wenn eulervm geladen!


%\usepackage{cm-super}
%% -------------------
%
%% - Times, Helvetica, Courier (Word Standard...)
%\usepackage{mathptmx}
%\usepackage[scaled=.90]{helvet}
%\usepackage{courier}
%% -------------------
%%
%% - Palatino , Helvetica, Courier


%%%%%%%%%%%%% Palatino mit Euler Math: Microtype: Letterspace +5
%\usepackage[sc]{mathpazo}   %% PW mit smallcaps-Option wird ein anderer Font mit besserem Kerning verwendet

%\usepackage[scaled=.95]{helvet}
%\usepackage{courier}

%\usepackage{tgpagella}  %%TeX-Gyre erweiterte Version der URW Palladio
%%%%The TeX Gyre Pagella family consists of 4 text fonts: regular,
%%%%italic, bold and bold italic (qplr, qplri, qplb, qplbi).
%%% Pagella sollte durch microtype zus�tzliche Laufweite bekommen!

%\usepackage{breqn}  %%PW: Nach allen Mathe-bezogenen Paketen laden. Experimentell! Teil des MH bundles

%% -------------------
%
%% - Bera Schriften
%\usepackage{bera}
%% -------------------
%
%% - Charter, Bera Sans
%Charter besser per mathdesign-bitstream
%\usepackage{charter}\linespread{1.05}    %Probleme mit Microtype
%\usepackage{charter}
%\renewcommand{\sfdefault}{fvs}


%% ===== Serifen =========================================================

%\usepackage{mathpazo}                 %% --- Palatino
%\usepackage{charter}\linespread{1.05} %% --- Charter
%\usepackage{bookman}                  %% --- Bookman (laedt Avant Garde !!)
%\usepackage{newcent}                  %% --- New Century Schoolbook (laedt Avant Garde !!)

%\usepackage[%                         %% --- Fourier
%%   upright,     % Math fonts are upright
%%   expert,      % Only for EXPERT Fonts!
%%   oldstyle,    % Only for EXPERT Fonts!
%%   fulloldstyle % Only for EXPERT Fonts!
%]{fourier} %  %%PW:gut, au�er mathcal und mathbb
%

%% Gentium
%%\usepackage{gentium}


%% ===== Sans Serif ======================================================

%\usepackage[scaled=.95]{helvet}        %% --- Helvetica
%\usepackage{cmbright}                  %% --- CM-Bright (eigntlich eine Familie)  %% Geht nicht mit Microtype!!
%\usepackage{hfbright} ?????                  %% --- CM-Bright (eigntlich eine Familie) als Type1  %% Geht erst durch hfbright mit Microtype!!
%\usepackage{tpslifonts}                %% --- tpslifonts % Font for Slides
%\usepackage{avantgar}                  %% --- Avantgarde


%% Arev, Sans-serif
%\usepackage{arev}

%% Iwona
%\usepackage[math]{iwona}
%\usepackage{iwona}
%%%%\renewcommand{\rmdefault}{lm}
%\renewcommand{\rmdefault}{iwona}
%\renewcommand*\familydefault{\sfdefault} %% Only if the base font of the document is to be sans serif

%% TeX Gyre Heros
%\usepackage{tgheros}


%% URW Optima, braucht Schriftdateien
%\renewcommand*\sfdefault{uop}
%\renewcommand*\familydefault{\sfdefault} %% Only if the base font of the document is to be sans serif

%Klappt nicht wirklich
%\usepackage{epigrafica}


%%%% =========== Italics ================

%\usepackage{chancery}                  %% --- Zapf Chancery

%%%% =========== Typewriter =============

%\usepackage{courier}                   %% --- Courier
%\renewcommand{\ttdefault}{cmtl}        %% --- CmBright Typewriter Font

%\renewcommand{\ttdefault}{lmtt}        %% --- Latin Modern Typewriter Font   %%PW: Hat Probleme bei Codelistings, bfseries sieht man nicht.
%\renewcommand{\ttdefault}{txtt}        %% --- Aus txfonts (Times-Paket)   %%ist auch ganz ok, aber nahe an den kpfonts
%\usepackage{inconsolata}   %%PW: Kann kein bfseries.

%%\usepackage[scaled]{beramono}

%\usepackage[%                          %% --- Luxi Mono (Typewriter)
%   scaled=0.9%
%]{luximono}

%%\usepackage{inconsolata}  %% nicht fett und nicht kursiv!



%%%% =========== Mathe ================


%%%\renewcommand{\rmdefault}{ugm} %nicht n�tig PW. Urw GaraMond
%
%\usepackage[
%% %   adobe-utopia,
%%   garamond,
%   bitstream-charter   %% so wird Charter geladen!  %%Problem mit Unendlich-Zeichen, wenn eulervm geladen
% ]{mathdesign}


%\renewcommand{\rmdefault}{bch}		%%PW bitstream charter

%%%\renewcommand{\sfdefault}{lmss}   %%PW latin modern sans serif

%%%% (((( !!! kommerzielle Schriften !!! )))))))))))))))))))))))))))))))))))))))))))))))))))

%% ===== Serifen (kommerzielle Schriften ) ================================

%% --- Adobe Aldus
%\renewcommand{\rmdefault}{pasx}
%\renewcommand{\rmdefault}{pasj} %%oldstyle digits
% math recommended: \usepackage[small]{eulervm}

%% --- Adobe Garamond
%\usepackage[%
%   osf,        % oldstyle digits
%   scaled=1.05 %appropriate in many cases
%]{xagaramon}
% math recommended: \usepackage{eulervm}

%% --- Adobe Stempel Garamond
%\renewcommand{\rmdefault}{pegx}
%\renewcommand{\rmdefault}{pegj} %%oldstyle digits

%% --- Adobe Melior
%\renewcommand{\rmdefault}{pml}
% math recommended: %\usepackage{eulervm}

%% --- Adobe Minion
%\renewcommand{\rmdefault}{pmnx}
%\renewcommand{\rmdefault}{pmnj} %oldstyle digits
% math recommended: \usepackage[small]{eulervm} or \usepackage{mathpmnt} % commercial

%% --- Adobe Sabon
%\renewcommand{\rmdefault}{psbx}
%\renewcommand{\rmdefault}{psbj} %oldstyle digits
% math recommended: \usepackage{eulervm}

%% --- Adobe Times
% math recommended: \usepackage{mathptmx} % load first !
%\renewcommand{\rmdefault}{ptmx}
%\renewcommand{\rmdefault}{ptmj} %oldstyle digits

%% --- Linotype ITC Charter
%\renewcommand{\rmdefault}{lch}

%% --- Linotype Meridien
%\renewcommand{\rmdefault}{lmd}

%%% ===== Sans Serif (kommerzielle Schriften) ============================

%% --- Adobe Frutiger
%\usepackage[
%   scaled=0.90
%]{frutiger}

%% --- Adobe Futura (=Linotype FuturaLT) : Sans Serif
%\usepackage[
%   scaled=0.94  % appropriate in many cases
%]{futura}

%% --- Adobe Gill Sans : Sans Serif
%\usepackage{gillsans}

%% -- Adobe Myriad  : Sans Serif
%\renewcommand{\sfdefault}{pmy}
%\renewcommand{\sfdefault}{pmyc} %% condensed Font

%% --- Syntax : sans serif font
%\usepackage[
%   scaled
%]{asyntax}

%% --- Adobe Optima : Semi Sans Serif
%\usepackage[
%   medium %darker medium weight fonts
%]{optima}

%% --- Linotype ITC Officina Sans
%\renewcommand{\sfdefault}{lo9}



