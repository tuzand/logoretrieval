\chapter{Proposal Based Object Detection and Classification}

\section{Fully Convolutional Neural Networks}
Long et. al. proposed \cite{DBLP:journals/corr/LongSD14} how to transform a deep neural network with fully connected classifier layers at the end, to a fully convolutional network. For this purpose the fully connected layers are to be removed from the end of the network.
Fully connected layers need a fix sized input. Since a fully convolutional network does not have fully connected layer anymore, it has the advantage of being able to train and test with images of arbitrary sizes.
The output of such a network are two dimensional feature maps, which can be used as heatmaps per class. These convolutional maps can also be used directly for semantic segmentation, where each pixel of an image should be classified.

\section{Region proposal systems}
Region proposals are possible object locations on the image. As we saw in the chapter two, the multiscale sliding window method induces a lot of computational costs. In order to reduce this computational burden, region proposal systems can be used. Earlier solutions used external proposal systems, e.g. Selective Search \cite{Uijlings13}. Today the proposal system is already part of the neural network.

\section{Region Based Convolutional Neural Network}
Region based convolutional neural networks use external region proposals. The network is used to run the complete inference on every region proposal bounding boxes. 

\section{Fast Region Based Convolutional Neural Network}
This type of network \cite{Girshick:2016:RCN:2881668.2882239} consists of a fully connected neural network, also called as base network in \citep{journals/corr/SermanetEZMFL13}, and a fully connected classifiation network. This version of region based convolutional neural network also uses external region proposals. First the image is completely pushed through the fully convolutional network. The output is a downscaled feature map, which is fed to the so called ROI Pooling layer. This layer crops regions from the map according to the appropriate downscaled region proposals, and executes pooling on each regions. After the pooling there are some fully connected layers, which make the final classification of the regions of interest. The advantage of this method is the much shorter inference time, achieved by the much lower computational redundancy of running convolutional layers on the whole image only once.

\section{Faster Region Based Convolutional Neural Network}
\cite{NIPS2015_5638}