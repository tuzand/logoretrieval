\chapter{Zitierungen, Randbilder}

Verweis auf etwas auf einer anderen Seite mit dem \texttt{varioref}-Paket: Siehe im Bildkapitel \vref{chap:Images}. Wenn das auf einer anderen Seite ist, wird die Seite mit angegeben.
Auf der gleichen Seite bei Abbildung \vref{exPDFEPSwdh} wird keine Seite angegeben, au�er man erzwingt es. Abbildung \vpageref{exPDFEPSwdh2} und Abbildung \vpageref{exPDFEPSwdh2}. Ist nur eine Seite unterschied wird das in Worten ausgeschrieben.

Verweis auf etwas auf einer anderen Seite mit dem \texttt{cleveref}-Paket: Siehe im Bildkapitel \cref{chap:Images}.

Very important work has been done by Roldan \cite{Roldan1997,Roldan1998,Roldan2001,Roldan2001a,Roldan2003,Roldan2005}.

	\begin{equation}
		\boxed{	\left\langle V \right\rangle = - \frac{2}{3 k_B T} \left(\frac{\mu_1 \mu_2}{4 \pi \epsilon_0 R^3}\right)^2 }
	\end{equation}

Mal schauen, was der Randbild-Befehl mit den normalen Bildern macht. Zuerst also ein tolles Bild:
\begin{figure}[htb]
\Centering
\includegraphics[scale=0.4]{images/EPStoPDF_Bild.pdf}
\caption[Captionformattest]{Nochmal ein Bild vom Anfang um zu sehen ob das Captionformat beim Randbild was ausmacht.}
\label{exPDFEPSwdh}
\end{figure}
 
Ein Beispiel f�r die Unterscheidung zwischen der Basis und dem Gitter liefert uns der NaCl-Kristall (herk�mmliches Kochsalz).  Man kann die Kristallstruktur von NaCl aufbauen, indem man abwechselnd Na$^+$- und Cl$^-$-Ionen auf die Gitterpunkte eines einfachen kubischen Gitters setzt. (siehe Abbildung Gitterstrukturen). Um Promille auszudr�cken kann man \textperthousand benutzen (aus dem \texttt{textcomp}-Paket) oder eine h�sslichere Variante \permil (aus \texttt{wasysym})

%Referenz aufs erste Bild \ref{keinnacl1}. Und jetzt das Randbild im Code.

%\randbild{images/bcc2.eps}{Kein NaCl Kristall}{Kein NaCl Kristall}{0.8}{keinnacl1}  % Bild-Pfadname, Kurzbeschriftung (f�r Abbildungsverzeichnis), Beschriftung, Breite zw. 0 und 1 = 100% des Randes, Label Referenz

In diesem Kristall ist jedes Ion umgeben von sechs n�chsten Nachbarn entgegengesetzter Ladung. Das Raumgitter des NaCl ist jedoch nach unserer Definiton kubisch fl�chenzentriert mit einer zweiatomigen Basis, eben einem Na$^+$- und einem Cl$^-$-Ion. Die Kristallstruktur ergibt sich dann sozusagen durch das ineinanderstellen der jeweils kubisch fl�chenzentrierten Gitter der beiden Ionensorten.

In %
%\marginnote{%
%\rule{1.5cm}{1cm}%
%\captionof{figure}[caption f�r toc]{Meine Bildunterschrift}%
%}%
diesem Kristall ist jedes Ion umgeben von sechs n�chsten Nachbarn entgegengesetzter Ladung. Das Raumgitter des NaCl ist jedoch nach unserer Definiton kubisch fl�chenzentriert mit einer 
%\marg{images/bcc.eps}
zweiatomigen Basis, eben einem Na$^+$- und einem Cl$^-$-Ion. Die Kristallstruktur ergibt sich dann sozusagen durch das ineinanderstellen der jeweils kubisch fl�chenzentrierten Gitter der beiden Ionensorten.

In diesem Kristall ist jedes Ion umgeben von sechs n�chsten Nachbarn entgegengesetzter Ladung. Das Raumgitter des NaCl ist jedoch nach unserer Definiton kubisch fl�chenzentriert mit einer zweiatomigen Basis, eben einem Na$^+$- und einem Cl$^-$-Ion. Die Kristallstruktur ergibt sich dann sozusagen durch das ineinanderstellen der jeweils kubisch fl�chenzentrierten Gitter der beiden Ionensorten.

\npar
Die Punkte des Gitters werden durch Gittervektoren\index{Gittervektoren} miteinander verbunden. An den Gitterpunkten selbst mu� kein Atom vorliegen, sie sind nur Punkte der Periodizit�t.

%Referenz auf das zweite Bild \ref{keinnacl2}, aber auch nochmal aufs erste \ref{keinnacl1}

%\randbild{images/bcc2.eps}{Immer noch kein NaCl Kristall}{Immer noch kein NaCl Kristall}{0.8}{keinnacl2}  % Bild-Pfadname, Kurzbeschriftung (f�r Abbildungsverzeichnis), Beschriftung, Breite zw. 0 und 1 =

%% Ulrike Fischer Beispiel
\begin{figure}
\Centering
	\marginnote{Legende\\ erster Eintrag}%
	\rule{3cm}{2cm} %Box, Breite, H�he
	\caption[caption f�r toc]{\marginnote{Legende\\ erster Eintrag}Bildunterschrift. Ja, das ist mit Absicht ein schwarzer Block.}
	\label{fig:Bild}
\end{figure}


\begin{figure}[htb]
\Centering
\includegraphics[scale=0.4]{images/EPStoPDF_Bild.pdf}
\caption[Captionformattest]{Nochmal ein Bild vom Anfang um zu sehen ob das Captionformat beim Randbild was ausmacht.}
\label{exPDFEPSwdh2}
\end{figure}

\blindtext

I'd like to cite \cite{Mahmoudi2009} as an underscore URL test, \cite{Agrawal} \cite{Agarwal88} and especially \cite{Bludau98} and \cite{Barmenkov2004}.
Not to forget \cite{Pask95} and the work of \cite{Paschotta97}.

\blindtext

Very nice are the papers of \cite{Mahmoudi2009,Voigt2005,DWandt1997,Yeh,Voigt2004}.

%\bibliographystyle{bib/bst/AlphaDINFirstName}
%\bibliography{bib/BibtexDatabase}