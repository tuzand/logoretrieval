\chapter{Introduction}

Advertising through static logos is one of the most important marketing method. A very effective possible way to reach a lot of people with these static logos, to sponsor sport teams or to buy advertising spaces in sport events. However the prices of these surfaces defined by the broadcasting companies mean huge expenses for the companies. The calculation of the costs is based firstly on the proportional size of the logo to the screen, and secondly on the time one particular logo is visible on the screen. This is the reason why the need for logo appearance statistics of sport videos arises. In particular there is a desire for quantitative measurement of these data to pay only for the provided service.

In this work I present a system for logo retrieval with proposal based object detection and classification. The system consists of a logo detector, and a classificator used for feature extraction. The logo detector is a faster region based convolutional neural network trained to recognise logos on images. The features of the proposed regions are extracted with a ResNet neural network. To recognise logos in videos, I will cut the videos into frames, and run the system for every image.

The challenge of this task is manifold. First problem is that the logos in these videos are far from being perfectly clear. They can be partially occluded, blurred - if the camera is moving fast, perspectively transformed, rotated and can have various coloring, suiting well to the design of the shirt or the arena. In addition there is a problem of the ambient illuminance variation just as for other computer vision tasks.
Second challenge is that however the logo detector is trained to recognise only one class, different brand's logos have a huge appearance variation. This makes the detection of logos very challenging.
Furthermore there are only a few smaller publicly available datasets, with bounding box annotated logos.

In the decade before last, hand-crafted feature extraction was prevalent in computer vision tasks. It needed an expert and often did not yield so good results. Deep learning methods for computer vision problems are dominant since the success of convolutional neural networks in 2012 [VGG]. This development is mainly powered by the annually organised ImageNet classification challenge (ILSVRC). Since the aim of this contest to classify an object, which is filling out the majority of an image, the location of the particular object does not play a specific role. To be able to classify and recognise objects, which have a much smaller size relative to the size of the whole image, region based classification can be utilised.

The rest of this thesis is organized as follows. Section 2 reviews the related work within image retrieval, object detection and logo retrieval. In Section 3 I will introduce the proposal based object detection with convolutional neural networks. In Section 4 I describe the logo retrieval system. Afterwards in Section 5 I evaluate and compare the system with another logo retrieval method. Finally I conclude my work and give prospects on future work in Section 6.