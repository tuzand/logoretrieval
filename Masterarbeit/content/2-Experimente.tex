\chapter{Experimente}

\blindtext


Ich referenziere das Nonfloatlisting \ref{nonfloatlisting}, das Floatlisting \ref{floatlisting}, das Mordslemma \ref{mordslemma} und den �bersatz \ref{ubersatz}. Da das erste Listing nicht in einer Flie�umgebung (float) liegt, kann es umbrochen werden. Das wird man aber nur im Einzelfall wirklich wollen und muss es selbst entscheiden.


\section{Listings}

\begin{lstlisting}[captionpos=b,label=nonfloatlisting,caption=A non-floating example]
for(int i=0; i<array.size(); ++i)
{
	std::cout << "Do nothing!\n"; 	//do nothing
}
\end{lstlisting}

\blindtext

\begin{lstlisting}[float,label=floatlisting,captionpos=b,caption=A floating example without line numbers,style=nonumbers]
for(size_t i=0; i<array.size(); ++i)
{
	std::cout << "Do nothing!\n"; 	//do nothing
}
\end{lstlisting}

\blindtext

Und noch ein Satz zum �bergang. (Abstand arg klein.)

\section{Theoreme}

\begin{Lemma}
\label{mordslemma}
Das ist mal ein M�rderlemma. Das ist mal ein M�rderlemma. Das ist mal ein M�rderlemma. Das ist mal ein M�rderlemma. Das ist mal ein M�rderlemma. 
\end{Lemma}

\begin{Satz}
\label{ubersatz}
Das ist mal ein �ber-Satz. Das ist mal ein �ber-Satz. Das ist mal ein �ber-Satz. Das ist mal ein �ber-Satz. Das ist mal ein �ber-Satz. 
	\begin{Beweis}
		Und das ist der Beweis daf�r!! Und das ist der Beweis daf�r!! Und das ist der Beweis daf�r!! Und das ist der Beweis daf�r!! 
	\end{Beweis}
\end{Satz}

\begin{Lemma}
Noch so ein Wahnsinnsteil! Noch so ein Wahnsinnsteil! Noch so ein Wahnsinnsteil! Noch so ein Wahnsinnsteil! Noch so ein Wahnsinnsteil! Noch so ein Wahnsinnsteil! Noch so ein Wahnsinnsteil! Noch so ein Wahnsinnsteil! Noch so ein Wahnsinnsteil! Noch so ein Wahnsinnsteil! Noch so ein Wahnsinnsteil!
\end{Lemma}


\begin{Beispiel}
Und das ist jetzt ein Beispiel: Und das ist jetzt ein Beispiel: Und das ist jetzt ein Beispiel: Und das ist jetzt ein Beispiel: Und das ist jetzt ein Beispiel: Und das ist jetzt ein Beispiel: Und das ist jetzt ein Beispiel: Und das ist jetzt ein Beispiel: Und das ist jetzt ein Beispiel: 
\end{Beispiel}

\begin{Example}
And this is an English example. And this is an English example. And this is an English example. And this is an English example. And this is an English example.
\end{Example}

\begin{Theorem}
This is an English theorem. This is an English theorem. This is an English theorem. This is an English theorem.\\
This is an English theorem. This is an English theorem. This is an English theorem. 
	\begin{Proof}
	And we can proof that. And we can proof that. And we can proof that. And we can proof that. And we can proof that. And we can proof that. 
	\end{Proof}
\end{Theorem}

\begin{Theorem}
This is an English theorem. This is an English theorem.\\
This is an English theorem. This is an English theorem. This is an English theorem. This is an English theorem. This is an English theorem. 
	\begin{Proof}
	And we can proof that. And we can proof that. And we can proof that. And we can proof that. And we can proof that. And we can proof that. 
	\end{Proof}
\end{Theorem}

\begin{Lemma}
Noch so ein Wahnsinnsteil! Noch so ein Wahnsinnsteil! Noch so ein Wahnsinnsteil! Noch so ein Wahnsinnsteil! Noch so ein Wahnsinnsteil! Noch so ein Wahnsinnsteil! Noch so ein Wahnsinnsteil! Noch so ein Wahnsinnsteil! Noch so ein Wahnsinnsteil! Noch so ein Wahnsinnsteil! Noch so ein Wahnsinnsteil!
\end{Lemma}

Es gibt:
\begin{itemize}
	\item Theorems: Theorem, Lemma, Proposition, Corollary, Satz, Korollar,
	\item Definitions: Definition,
	\item Examples: Example, Beispiel,
	\item Remarks: Anmerkung, Bemerkung, Remark,
	\item Proofs: Proof and Beweis
\end{itemize}

Very nice are the papers of \cite{Mahmoudi2009,Voigt2005,DWandt1997,Yeh,Voigt2004}.

%\bibliographystyle{bib/bst/AlphaDINFirstName}
%\bibliography{bib/BibtexDatabase}