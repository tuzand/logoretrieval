\chapter{Related Work}\label{c:relatedwork}

This chapter gives an outline on the recent research related to this work. The included topics are deep learning, object detection, image retrieval, open-set classification and logo retrieval.
Deep neural networks are motivated by the memorization procedure of the human brain. Rosenblatt invented perceptrons \cite{Rosenblatt58theperceptron:}, which serve as the basis of today's Fully Connected layers (FC) in deep learning. LeCun applied \cite{LeCun:1989:BAH:1351079.1351090} Convolutional Neural Networks (CNN) for handwritten digits recognition in 1989, which are inspired by the visual cortex of the nervous system. Since then, a lot of research was made to improve and extend the application scope of deep learning in computer vision.
\section{Object Detection}
Earlier systems utilized hand-crafted features to detect objects on images and recognize them. Lowe et.al., used Scale and Translation Invariant Features (SIFT) \cite{Lowe:2004:DIF:993451.996342} around keypoints, detected with e.g. Harris corner detector \cite{Harris}. Viola and Jones utilized \cite{Viola:2004:RRF:966432.966458} Haar-like features and a cascade of weak classifiers (Adaptive Boosting \cite{Schapire:1999:BIB:1624312.1624417}) to detect faces extremely fast. Nowadays, deep learning methods surpass the traditional methods by a wide margin. OverFeat framework \cite{journals/corr/SermanetEZMFL13} uses sliding windows on multiple scales of the image, and combines the features to detect objects and to classify them. You Only Look Once (YOLO) \cite{DBLP:journals/corr/RedmonDGF15} introduces an end-to-end network for detecting and classifying objects by using bounding box regressors at the first time for localization. It splits the input image into a square grid, where every cell predicts several bounding boxes with probability scores and classification labels. The Single Shot MultiBox Detector (SSD) \cite{journals/corr/LiuAESR15} utilizes convolutional features from multiple layers and concatenates them to detect objects in real time. Faster Region-Based Convolutional Neural Network (R-CNN) \cite{NIPS2015_5638} will be detailed in chapter \ref{c:theory}. Region-based Fully Convolutional Network (R-FCN) \cite{DBLP:journals/corr/DaiLHS16} is the improvement of Faster R-CNN in terms of inference time by having a network end-to-end fully convolutional. Mask R-CNN \cite{he2017maskrcnn} extends the functionality of Faster R-CNN by extending the network with a classification mask, which allows end-to-end object detection and semantic segmentation with a little overhead.
\section{Image Retrieval}
Many techniques exist outside the scope of deep learning for image retrieval from videos. SIFT features \cite{Lowe:2004:DIF:993451.996342} with bag-of-visual-words were used to get translation invariant descriptors around keypoints by Zisserman \cite{Sivic:2003:VGT:946247.946751}. The visual words were then used to retrieve objects in videos instantaneously like searching on Google. Histogram of Oriented Gradients (HOG) \cite{Dalal:2005:HOG:1068507.1069007} descriptors are gradient orientation histograms, extracted blockwise as features. Hu et.al. \cite{Hu:2013:PEG:2479988.2480107} used an extended version of HOG descriptors to retrieve images based on sketches from a database. Nowadays, deep learning approaches are prevalent in the context of image retrieval. In \cite{Yan:2016:CVS:2964284.2967252}, CNN and SIFT features were compared and fused to retrieve images. Babenko et.al. \cite{DBLP:journals/corr/BabenkoSCL14} utilized the output of the middle fully connected layer to retrieve images taken in same or similar scenes. This approach is used also in this work, however Babenko used the complete image as input, not low resolution Region of Interests (RoIs). Gordo et.al. \cite{DBLP:journals/corr/GordoARL16} extracted local features from different regions of an image, which are selected by a region proposal system, and combined them to a global feature to retrieve images with similar scenes. In \cite{DBLP:journals/corr/RadenovicTC16} Maximum Activations of Convolutions (MAC) was computed from the output of a Fully Convolutional Network to represent images. MAC is a max pooling from the two dimensions of each channels, then the max value of every channel is used as a feature.
\section{Open Set Classification}
Bendale et.al. argue \cite{DBLP:journals/corr/BendaleB15} with the fact that the majority of proposed neural networks utilize softmax layer to get classification probabilities. Thus, these networks are inherently trained for a closed-set classification world. These models can easily be fooled with abstract images, humans easily reject from any of the trained classes, but with the network predicting a class label with high probability. For this purpose, they propose OpenMax, which can be used to replace the softmax layer. OpenMax is able to classify such fooling and adversarial images as unknown.
\section{Logo Retrieval}
Romberg et.at. \cite{Romberg:2011:SLR:1991996.1992021} published FlickrLogos-32 dataset, which became the standard evaluation dataset of logo retrieval systems. Furthermore, Romberg et.al. \cite{conf/mir/RombergL13} generated synthetic data to increase the training dataset size, and combined local features from logo images into an aggregated feature. Pandey et.al. \cite{DBLP:conf/icip/PandeyDJPB14} used SIFT features and bag-of visual words to retrieve logos from natural images, where the logo filled the complete image. In \cite{DBLP:journals/corr/HoiWLWWXW15} a large logo dataset of 100 brands with about 130,000 instances was introduced, which is unfortunately still not publicly available. In \cite{Eggert:2015:BSD:2733373.2806407}, synthetic logo dataset was utilized and a neural network was trained with unlabeled data by bootstrapping. Iandola et.al. \cite{DBLP:journals/corr/IandolaSGK15} used Fast R-CNN for the first time to retrieve logos from images. Furthermore, R-CNN, Fast R-CNN and Faster R-CNN were used in \cite{Bao:2016:RCL:3007669.3007728}, \cite{DBLP:journals/corr/OliveiraFPR16}, \cite{DBLP:journals/spl/QiSWX17}. In \cite{DBLP:journals/corr/SuZG16} synthetic logo data was used to extend a train dataset with very scarce size. In 2017 Bianco et.al. introduced Logos-32Plus \cite{bianco2017deep}, which is currently the largest publicly available dataset with 12,312 RoIs altogether.