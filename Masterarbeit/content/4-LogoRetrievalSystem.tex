\chapter{Logo Retrieval System}

\section{Logo Datasets}
The hunger of deep learning method for training data is well-known. As the publicly available logo datasets are quite small, a better training result can be achieved if the datasets are merged. The different datasets with the number of brands, images and bounding box rois can be seen in table \ref{table:logodatasets}. The total number of brands means the number of different brands altogether.

There are also trademark datasets available having a much greater cardinality \cite{DBLP:journals/corr/TursunAK17}. The images of this dataset contain however only the logo of a company, without any context of the logo. This dataset turned out to have no use for region based deep learning methods, since this approach needs to learn to distinguish between objects to be learned and the background. The network was trained with the fusion of FlickrLogos-32 and the trademark dataset, and tested with the evaluation method of FlickrLogos-32.
\begin{table}[ht!]
\centering
\caption{Publicly available logo datasets with with bounding box annotations}
\label{table:logodatasets}
\begin{tabular}{|l|l|l|l|}
\hline & \textbf{Number of brands} & \textbf{Number of logo images} & \textbf{Number of ROIs} \\
\hline
\textbf{BelgaLogos} & 37 & 1321 & 2697 \\
\hline
\textbf{Flickr Logos 27} & 27 & 810 & 1261 \\
\hline
\textbf{FlickrLogos-32} & 32 & $70 \cdot 32 = 2240$ & 3404 \\
\hline
\textbf{Logos-32plus} & 32 & 7830 & 12300 \\
\hline
\textbf{TopLogo10} & 10 & $10 \cdot 70 = 700$ & 863 \\
\hline\hline
\textbf{Total (union)} & \textbf{80} & 12 901 & 20 525 \\ \hline
\end{tabular}
\end{table}

\section{Logo Detection}

\section{Logo Comparison}
