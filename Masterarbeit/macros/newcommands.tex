% Latex Commands: Math **********************************
% ------------------------------------------------------------------------
% Partly by: Matthias Pospiech
%%%%%%%%%%%%%%%%%%%%%%%%%%%%%%%%%%%%%%%%%%%%%%%%%%%%%%%%%%%%%%%%%%%%%%%%%%


%%PW: Aus nomencl-Handbuch zur Unterteilung des Symbolverzeichnisses
\RequirePackage{ifthen}
\renewcommand{\nomgroup}[1]{%
\ifthenelse{\equal{#1}{R}}{\item[\textbf{R�mische Zeichen}]}{%
\ifthenelse{\equal{#1}{G}}{\item[\textbf{Griechische Zeichen}]}{%
\ifthenelse{\equal{#1}{C}}{\item[\textbf{Kalligrafische Zeichen}]}{%
}%
}%
}%
}%


% --| Math |-------------------------------------------------------


%%%%%%%% Macht Probleme: Kompilierung h�ngt dann.
%%%% PW: Matrix elements alignment
%%%% Optionaler Parameter, default-Wert ist 'c'.
%%%% Erweitert nur den matrix-Befehl. Ohne das optionale Argument klappts auch noch.
%\renewcommand*\env@matrix[1][c]{%
%\hskip -\arraycolsep%
%\let\@ifnextchar\new@ifnextchar%
%\array{*\c@MaxMatrixCols #1}%
%}


% -- Replacements --
\newcommand{\comp}{\ast}
%\renewcommand{\dagger}{+}

% -- new commands --
\providecommand{\abs}[1]{\lvert#1\rvert}
\providecommand{\Abs}[1]{\left\lvert#1\right\rvert}
\providecommand{\norm}[1]{\left\Vert#1\right\Vert}
\providecommand{\Trace}[1]{\ensuremath{\Tr\{\,#1\,\}}} % Trace /Spur
%

% -- differentials --
\renewcommand{\d}{\partial\mspace{2mu}} % partial diff
\newcommand{\td}{\,\mathrm{d}}	% total diff
\newcommand{\ddt}[1]{\frac{\td #1}{\td t}}

% -- Abbrevitations --
%\renewcommand{\Re}{\text{Re}}			% Real value
%\renewcommand{\Im}{\text{Im}}			% Imaginary value
\newcommand{\Rep}{\text{Re}}			% Real value
\newcommand{\Imp}{\text{Im}}			% Imaginary value
\newcommand{\complex}{\mathbb{C}} % Complex
\newcommand{\real}{\mathbb{R}}    % Real
%\newcommand{\R}{\real}						% Real
%\newcommand{\N}{\mathbb{N}}
%\newcommand{\Z}{\mathbb{Z}}
\renewcommand{\L}{\mathcal{L}}
%%\newcommand{\N}{\mathcal{N}}
%%\newcommand{\R}{\mathcal{R}}
\newcommand{\Dc}{\mathcal{D}}
%%PW besser so:
 \newcommand{\C}{\ensuremath{\mathbb{C\ }}}
 \newcommand{\R}{\ensuremath{\mathbb{R\ }}}
 \newcommand{\Q}{\ensuremath{\mathbb{Q\ }}}
 \newcommand{\Z}{\ensuremath{\mathbb{Z\ }}}
 \newcommand{\N}{\ensuremath{\mathbb{N\ }}}
 
 \newcommand{\Calt}{\ensuremath{\mathds{C\ }}}
 \newcommand{\Ralt}{\ensuremath{\mathds{R\ }}}
 \newcommand{\Qalt}{\ensuremath{\mathds{Q\ }}}
 \newcommand{\Zalt}{\ensuremath{\mathds{Z\ }}}
 \newcommand{\Nalt}{\ensuremath{\mathds{N\ }}}


%%PW
%%Wenn Paket bm nicht geladen ist, erste Zeile einkommentieren
%\newcommand{\bm}[1]{\ensuremath{\boldsymbol{#1}}}
\newcommand{\mb}[1]{\ensuremath{\bm{#1}}}

%
\newcommand{\Ham}{\mathcal{H}}    % Hamilton
%\newcommand{\Prob}{\mathscr{P}}    % Hamilton
\newcommand{\unity}{\mathds{1}}   % Real
%

\newcommand\gammab{\gamma_\bot}
\newcommand\gammap{\gamma_\parallel}
\newcommand\gammai{\gamma_\text{int}}
\newcommand\gammae{\gamma_\text{ext}}


\newcommand{\ie}{i.e.}
\newcommand{\eg}{e.g.}
\newcommand{\dhe}{d.\,h.\ }
\newcommand{\Dhe}{D.\,h.\ }
\newcommand{\eV}{e.\,V.\ }
\newcommand{\zB}{z.\,B.\ }
\newcommand{\zb}{z.\,B.\ }
\newcommand{\oae}{o.\,\"{a}.\ }
\newcommand{\uU}{u.\,U.\ }
\newcommand{\ua}{u.\,a.\ }
\newcommand{\ia}{i.\,a.\ }
\newcommand{\og}{o.\,g.\ }
\newcommand{\bzw}{bzw.\ }
\newcommand{\usw}{usw.\ }
\newcommand{\vgl}{vgl.\ }
\newcommand{\etc}{etc.\ }
\newcommand{\evtl}{evtl.\ }
\newcommand{\ca}{ca.\ }
\newcommand{\sog}{sog.\ }
\newcommand{\insbes}{insbes.\ }

\newcommand{\datei}[1]{\texttt{\bfseries #1}}


%PW: Evtl. n�tzliche Macros

\DeclareMathOperator*{\argmax}{arg\,max}
\DeclareMathOperator*{\argmin}{arg\,min}

\newcommand{\Vektor}[1]{\bm{#1}}
\newcommand{\Matrix}[1]{\bm{#1}}
\newcommand{\Vek}[1]{\boldsymbol{#1}}
\newcommand{\NormVert}[2]{\mathcal{N}(#1,#2)}
\newcommand{\Pb}[1]{\ensuremath{p \left( #1 \right)}}
\newcommand{\Prob}[2]{\ensuremath{p \left( #1 \middle\vert #2 \right)}}
\newcommand{\SkalProd}[2]{\langle #1,#2 \rangle}
\newcommand{\SkalProdD}[2]{\left\langle #1,#2 \right\rangle}

\newcommand{\Complexity}{\ensuremath{\mathcal{O}}}

\newcommand{\diftwo}{\ensuremath{\dif\hspace{0.2pt}^{2}}}
\newcommand{\Menge}[1]{\ensuremath{\left\{#1\right\}}}

\newcommand*\expe{\mathrm{e}}


% Die Definition eigener mathematischer Befehle ist besonders sinnvoll, wenn diese im Dokument oft verwendet werden.
% Man kann dann hier an zentraler Stelle z.B. Vektoren mit Pfeil statt fett formatieren.

\newcommand{\supp}[1]{ \text{supp}\!\left\{#1\right\} }
\DeclareMathOperator{\falttwod}{\ast\ast} %
\DeclareMathOperator{\sollsein}{\;\stackrel{!}{=}\;} %

%%%% PW: Korrekte Initialisierung des widehat-Dach-Symbols
\newcommand\lowerwidehatsym{%
  \text{\smash{\raisebox{-1.3ex}{%
    $\widehatsym$}}}}
		
\newcommand\fixwidehat[1]{%
  \mathchoice
    {\accentset{\displaystyle\lowerwidehatsym}{#1}}
    {\accentset{\textstyle\lowerwidehatsym}{#1}}
    {\accentset{\scriptstyle\lowerwidehatsym}{#1}}
    {\accentset{\scriptscriptstyle\lowerwidehatsym}{#1}}
}

%% Verwendung in eigenen Macros z.B.:
%\newcommand{\PhiWithHatAndSuperscriptW}{\ensuremath{\fixwidehat{\Phi}^{\mathrm{W}}}}

%%%%%



% -- New Operators --
\DeclareMathOperator{\rot}{rot}
\DeclareMathOperator{\grad}{grad}
%\DeclareMathOperator{\div}{div}
\renewcommand{\div}{\text{div}\,}
\DeclareMathOperator{\Tr}{Tr}
\DeclareMathOperator{\const}{const}
\DeclareMathOperator{\e}{e} 			% exponatial Function

% -- new symbols --
\newcommand{\laplace}{\Delta}
\newcommand{\dalembert}{\Box}

% -- new arrows --
\renewcommand{\leadsto}{\Longrightarrow}
\newcommand{\leftrightleadsto}{\Longleftrightarrow}


% -- Text subscripts--
\newcommand{\rel}{_\text{rel}}
%\newcommand{\st}{\text{st}}
%

% -- other --
\newcommand{\com}[2]{\underbrace{#1}_{\textrm{\scriptsize #2}}}
\newcommand{\with}[1]{\stackrel{\ref{#1}}{\Longrightarrow}}
%\newcommand{\unit}[1]{\,\textrm{#1}}

%\newcommand{\variance}[1]{\delta \mean{#1}^2}
\newcommand{\variance}[1]{(\Delta{#1})^2}
%\newcommand{\variance}[1]{\delta #1^2}

% -- Physics --------------------------------
\newcommand\op[1]{{\hat{\mathrm{#1}}}}  % Operator

\newcommand\expect[1]{\ensuremath{\left\langle{#1}\right\rangle}} %
%
\newcommand{\mean}[1]{\ensuremath{\overline{#1}}} % mean value
%
\newcommand{\state}[1]{\ensuremath{\ket{#1}}}
%
\newcommand\commutator[2]{\ensuremath{\mathinner{%
    \mathopen[\,#1,#2\,\mathclose]}}}
\newcommand{\Commutator}[2]{\ensuremath{\left[\,#1,#2\,\right]}}
\newcommand{\bigcommutator}[2]{\ensuremath{\bigl[\,#1,#2\,\bigr]}}
\newcommand{\Bigcommutator}[2]{\ensuremath{\Bigl[\,#1,#2\,\Bigr]}}
%
\newcommand\poisson[2]{\mathinner{%
    \mathopen\{#1,#2\mathclose\}}}
%

% -- Layout --------------------------------



\newcommand*{\dashfill}{\leavevmode\cleaders\hbox{-}\hfill\kern0pt}

\newcommand*{\midhrulefill}{
\leavevmode
\cleaders\hbox to 1ex{\raisebox{.5ex}{\rule{1ex}{.4pt}}}\hfill\kern0pt
}



\newcommand{\naherung}{\marginlabel{\ovalbox{\sffamily{\small N�herung}}}} %N�herungszeichen


% PW: f�r bbordermatrix mit eckigen Klammern
% braucht etoolbox-Paket
% Alternativ: klassische plainTeX \bordermatrix mit eckigen Klammern (Redefinition von Befehlen):
\let\bbordermatrix\bordermatrix
\patchcmd{\bbordermatrix}{8.75}{4.75}{}{}
\patchcmd{\bbordermatrix}{\left(}{\left[}{}{}
\patchcmd{\bbordermatrix}{\right)}{\right]}{}{}


%% Set PDF properties with hyperref. nicht mit pdfx
%%\hypersetup{ pdftitle={\Worktitle}, pdfauthor={\Authorname}}

%%PW: l2tabu: Also \begin{center} macht in floats zus�tzlichen Abstand. \Centering ist richtig!

\newcommand{\bild}[6]      % Bild-Pfadname, Beschriftung, Label, Breite, Kurzbeschriftung (f�r Abbildungsverzeichnis), optional: Platzierung
{
     \begin{figure}[#6]
%      \begin{center}		%l2tabu
				\Centering
        \includegraphics*[width=#4]{#1}
        \caption[#5]{\label{#3} #2}
%      \end{center}			%l2tabu
     \end{figure}
}

%So funktioniert es nat�rlich auch noch. Captions sind schon besser. Der Captionname ist noch zu gro�
\newcommand{\randbild}[5]   % Bild-Pfadname, Kurzbeschriftung (f�r Abbildungsverzeichnis), Beschriftung, Breite zw. 0 und 1 = 100% des Randes, Label
{%
\marginpar%
	{%
	%%%%%\RaggedRight
	%%%%%\fontsize{5}{6.5}
	%%%%%\selectfont\selectfont
	\captionsetup{font=scriptsize,%footnotesize
	labelfont={bf,scriptsize},%footnotesize
	justification=raggedright, indention=0.0cm}  %Variante des Caption-Pakets. Inkompatibel zu nonfloat, wo figcaption herkommt.
		%\Centering% %% sorgt daf�r, dass Bild weiter links als Anfang der Caption. Mit weglassen nur noch schnlimmer
		\RaggedLeft  %% sorgt daf�r, dass Bild b�ndig mit Anfang der Caption
		\includegraphics[width=#4\marginparwidth]{#1}%
		\noindent
		%\setcapindent{-1em}  %KOMA-Script-Weg
		%\addtokomafont{caption}{\footnotesize}  %KOMA-Script-Weg
		%\setkomafont{caption}{\footnotesize\sffamily\linespread{1}\selectfont}
		%\setkomafont{captionlabel}{\footnotesize\sffamily}
		%\figcaption[#2]{\label{#5}{#3}}%
		%\parbox{#4\marginparwidth}
		%\figcaption[#2]{\label{#5}\footnotesize{#3}}%
		%\figcaption[#2]{\label{#5}\scriptsize{#3}}%
	}%
}



%%PW ---------------------- TikZ ------------------------
\usetikzlibrary{decorations.pathmorphing} % noisy shapes
\usetikzlibrary{fit}					% fitting shapes to coordinates
\usetikzlibrary{backgrounds}	% drawing the background after the foreground
\usetikzlibrary{arrows}
\usetikzlibrary{matrix,calc} % for bordermatrix example

%%%PW ------------------- Kommentare ------------------

\newenvironment{dacomment}%
{
\marginpar{\framebox{!}}%
\Centering%
\begin{minipage}{.85\textwidth}%
	\normalfont \mdseries \sffamily %
	\small \color{grau}%
\end{minipage}
}%
%Verwendung nicht mit \usepackage{version}, sondern mit comment



% % redefine \textmu to other mu commands usefull inside text
% \renewcommand{\textmu}{$\upmu$}


\makeatletter
%\renewcommand{\@chapcntformat}[1]{%
%	\fbox{\csname the#1\endcsname}%
%	\hspace{0.5em}}

%\renewcommand{\chapter}{\@startsection
%	{chapter}% 		% Name
%	{0}%								% ebene
%	{0mm}%							% einzug
%	{ - 0.5\baselineskip}% % Abstand nach oben
%	{0.2\baselineskip}% % Abstand nach unten
%	{\bfseries}}%				% Stil					

% Caption f�r nicht-flie�enden Figure Umgebung:
\newcommand\figcaption{\def\@captype{figure}\caption} 
\makeatother	

\newcommand\nz{\smallskip \\} %Neue Zeile
\newcommand\npar{\smallskip \par} %Neue Zeile

\newcommand{\figureref}[1]{(Abbildung \ref{#1})}%
\newcommand{\eqnref}[1]{(\ref{#1})}%

% Command for margin text with usefull style
%\newcommand{\marginlabel}[1]{\mbox{}\marginline{\hspace{0pt}\footnotesize\sffamily #1}}%
\newcommand{\marginlabel}[1]{\marginnote{#1}}%

%\newcommand{\comment}[1]{\marginnote{#1}}%

% Enable space for figures that extent into the margin (right and/or leftside)
% Can be used inside a figure
% Note: sidecap defines a similar environment 'wide' !
\newenvironment{widespace}[2]{%
   \begin{list}{}{%
      \setlength{\topsep}{0pt}%
      \setlength{\leftmargin}{#1}%
      \setlength{\rightmargin}{#2}%
      \setlength{\listparindent}{\parindent}%
      \setlength{\itemindent}{\parskip}%
   }%
   \item[]%
}%
{%
   \end{list}%
}%

\newlength{\marginwidth}
\setlength{\marginwidth}{\marginparwidth}
\addtolength{\marginwidth}{\marginparsep}

%% Beispiel:
% \begin{figure}
% \begin{widespace}{-\marginwidth}{0pt}
%  \subfloat[Bergzebrastute]
%  {\includegraphics[width=0.45\linewidth]{../Bilder/Eingewoehnung2.jpg}}
%  \hspace*{1em}
%  \subfloat[Morro Moco]
%  {\includegraphics[width=0.45\linewidth]{../Bilder/bergzebra2.jpg}}
% \end{widespace}
% \end{figure}




%% PW: Alternativ zu sidewaysfigure:
%% Erscheint im PDF (im Sumatra zumindest) bereits gedreht wenn pdflscape package statt lscape genommen wird.
%% Modifiziert von
%% http://tex.stackexchange.com/questions/62780/sideways-figure-with-rotated-page-number
%\newenvironment{LSfigure}[1][!t]
  %{\begin{landscape}\begin{figure}}
  %{\end{figure}\end{landscape}}


  







%So war es und so funktioniert es auch. Captions sind aber noch zu gro�
%\newcommand{\randbild}[5]   % Bild-Pfadname, Kurzbeschriftung (f�r Abbildungsverzeichnis), Beschriftung, Breite zw. 0 und 1 = 100% des Randes, Label
%{%
%\marginpar%
	%{%
		%\Centering%
		%\includegraphics[width=#4\marginparwidth]{#1}%
		%\figcaption[#2]{\label{#5}\small{#3}}%
	%}%
%}


%% PW
%% Aus http://mrunix.de/forums/showthread.php?t=69987
%% klappt auch nicht so wie es soll. Gibt eine normale Figure und einen "Floats lost" Fehler.
%\newcommand{\changefont}[3]{\fontfamily{#1}\fontseries{#2}\fontshape{#3}\selectfont}
%\newcommand{\marg}[1]{\marginnote{%
%\RaggedRight%
%\fontsize{9.5}{13.5}%
%\selectfont%
%\selectfont%
%\changefont{lmss}{b}{n}%
%\begin{figure}%
%\Centering%
%\includegraphics[width=\marginparwidth]{#1}%
%\caption[Testcaption]{Das ist eine Testcaption}%
%\label{TESTLABEL}%
%\end{figure}%
%}}



%% Das hier setzt die Caption v�llig falsch mittenrein.
%\newcommand{\randbild}[5]   % Bild-Pfadname, Kurzbeschriftung (f�r Abbildungsverzeichnis), Beschriftung, Breite zw. 0 und 1 = 100% des Randes, Label
%{%
%\marginnote{%
	%\captionsetup{font=footnotesize, labelfont={bf,footnotesize}, justification=raggedright, indention=0.0cm}  %Variante des Caption-Pakets. Inkompatibel zu nonfloat, wo figcaption herkommt.
		%\Centering%
		%\includegraphics[width=#4\marginparwidth]{#1}%
		%%\noindent
		%\captionof{figure}[#2]{\label{#5}{#3}}
	%}%
%}


%% PW: Nur Captions in den Rand, Bild bleibt im Text. Braucht Paket mcaption
%\newcommand{\randcapbild}[5]  % Bild-Pfadname, Kurzbeschriftung (f�r Abbildungsverzeichnis), Beschriftung, Breite zw. 0 und 1 = 100% des Randes, Label
%{%
%\begin{figure}
	%\begin{margincap}
		%\Centering
		%\includegraphics[width=#4\marginparwidth]{#1}
		%\caption[#2]{#3}
		%\label{#5}
	%\end{margincap}
%\end{figure}
%}





%%PW ----------------- TikZ Chapter headings ---------------------------------
%% aus http://texblog.net/latex-archive/layout/fancy-chapter-tikz/
%% Bringt TeX capacity exceeded, grouping levels






% Caption with defined width
\newcommand{\wcaption}[2]{%
   \begin{minipage}{#1}%
   \caption{#2}%
   \end{minipage}%
}