% -- new commands by MiG---

%Achtung! Diese Datei verwendet Hilfsmakros aus der Datei AuxiliaryMacros.tex!


%1.) Befehle zum Rumspielen mit Layout:

%Optionale Herleitungen, die bei Bedarf ausgelassen werden k�nnen:
%\newcommand{\reduce}[1]{\relax} % --> Optionale Herleitungen reduzieren
\newcommand{\reduce}[1]{#1} % --> Optionale Herleitungen nicht reduzieren

\newcommand{\code}[1]{\textsf{#1}}

%Befehl zum Hervorheben wichtiger Begriffe
%\newcommand{\mydef}[1]{\emph{#1}} %kursiv
%\newcommand{\mydef}[1]{\textbf{#1}} % fett
\newcommand{\mydef}[1]{\emph{\textbf{#1}}} %fett und kursiv

\newcommand{\rem}[1]{}  %Zum einf�gen von Kommentaren mitten in einer Zeile
%\newenvironment{rem}[1]{}{} %Blockkommentare funktionieren nicht, falls } mittendrin. 
% Verwende Umgebung {comment} aus dem entsprechendem Packet

%TODO-Befehle f�r den Textmodus
\newcommand{\todo}[1]{\myTodo{#1}} %\addToTodoList{#1}}
\newcommand{\ToDo}[1]{\todo{#1}}
\newcommand{\Todo}[1]{\todo{#1}}
\newcommand{\TODO}[1]{\todo{#1}}

%TODO-Befehle f�r den Mathe-Modus
\newcommand{\mathtodo}[1]{\myMathTodo{#1}} %\addToTodoList{$#1$}}
\newcommand{\mathToDo}[1]{\mathtodo{#1}}
\newcommand{\mathTodo}[1]{\mathtodo{#1}}
\newcommand{\mathTODO}[1]{\mathtodo{#1}}

%Erzeugen eines TODO-Bildes:
%[#1]: optional: Pfad zu einem bereits vorhandenen Bild, das �berarbeitet werden soll (in eckigen Klammern)
%{#2}: TODO-Beschreibung
\newcommand{\imgtodo}[2][]{\myImgTodo[#1]{#2}}
\newcommand{\imgToDo}[2][]{\myImgTodo[#1]{#2}}
\newcommand{\imgTodo}[2][]{\myImgTodo[#1]{#2}}
\newcommand{\imgTODO}[2][]{\myImgTodo[#1]{#2}}


%Add a ToDo without showing the TODO text in the document
\newcommand{\remtodo}[1]{\textcolor{red}{\textbf{ToDo!}}} %\addToTodoList{ToDo!}}


\newcommand<>{\mathred}[1]{{\color#2{red}#1}}
\newcommand<>{\mathblue}[1]{{\color#2{blue}#1}}