%  Sondersymbole:
%
%-------------------------------------------------------------------------------
%
%Objektbezogene Darstelung
%
% Beobachtbar:
\newcommand{\obs}{\supset\relax\relax}%
%
% Nicht beobachtbar:
\newcommand{\nobs}{\not\supset\relax\relax}%
%
%Phantom-Objekt:
%\newcommand{\phantomobj}{\mathrm{P}}%
\newcommand{\phantomobj}{\mathsf{P}}%
%
% Clutter:
\newcommand{\clutter}{\copyright}%
%\newcommand\clutter{\textcircled{c}}%
%
% Clutter source:
\newcommand{\cluttersource}{\copyright}%{\mathbf{x}_{\copyright}}%
\newcommand{\xc}{\mathbf{x}_{\copyright}}%
%
% New object
\newcommand{\birthsource}{\mathbf{x}_{B}}%
\newcommand{\xB}{\mathbf{x}_{B}}%
%
%
%
%
%-------------------------------------------------------------------------------
%
%Messung-bezogene Darstellung:
%
% Keine Messung
\newcommand{\nodet}{\varnothing}%\emptyset
\newcommand{\zMissing}{\mathbf{z}_{\varnothing}}%\emptyset
%
% Messung nicht sichtbar
\newcommand{\nobsmes}{\nobs}%
\newcommand{\zNobs}{\mathbf{z}_{\nobs}}%
%
% Messung nicht existent
\newcommand{\nexistsmes}{\nexists}%
\newcommand{\zNexists}{\mathbf{z}_{\nexists}}%
%
% Zusammenfassung von Verdeckungen und Sensorfehler in z_0:
% (Zusammenfassung der Nichtbeobachtbarkeit und Nicht-Sichtbarkeit)
% (Zusammenfassung der Nichtbeobachtbarkeit, Nicht-Sichtbarkeit und Nich-Existenz)
\newcommand{\nomes}{\mathbf{z}_{0}}
%
%
%-------------------------------------------------------------------------------
%
%Erwartungswert:
%
\newcommand{\E}{\mathbb{E}}
\newcommand{\Erw}[1]{ \E[#1] }
\newcommand{\ERW}[1]{ \E\bigl[#1\bigr] } %Mit gro�en Klammern
%
%
%-------------------------------------------------------------------------------
%
%Gauss-Verteilung an einer Stelle:
%
%\newcommand{\gauss}[3]{N({#1}; {#2}, {#3})}
\newcommand{\gauss}[3]{\N({#1}; {#2}, {#3})}
%Alte Schreibweise
%\newcommand{\gauss}[3]{\varphi_{{#2}, {#3}}({#1})}
%
%
%-------------------------------------------------------------------------------
%
\newcommand{\gate}{\Gamma}
%
%-------------------------------------------------------------------------------
%
%Zustand und Messung: Fettes x und z:
%
%(Realisationen):
\newcommand{\state}{\mathbf{x}}
\newcommand{\meas}{\mathbf{z}}
\newcommand{\x}{\mathbf{x}}
\newcommand{\z}{\mathbf{z}}
\newcommand{\xk}{\mathbf{x}_k}
\newcommand{\zk}{\mathbf{z}_k}
\newcommand{\xii}{\mathbf{x}_i}
\newcommand{\zj}{\mathbf{z}_j}
\newcommand{\uk}{\mathbf{u}_k}
\newcommand{\xki}{\mathbf{x}_{k,i}}
\newcommand{\zkj}{\mathbf{z}_{k,j}}
%
%(Zufallsvariablen):
\newcommand{\Xk}{\mathbf{X}_k}
\newcommand{\Zk}{\mathbf{Z}_k}
\newcommand{\Uk}{\mathbf{U}_k}
\newcommand{\Wk}{\mathbf{W}_k}
\newcommand{\Vk}{\mathbf{V}_k}
\newcommand{\Xki}{\mathbf{X}_{k,i}}
\newcommand{\Zkj}{\mathbf{Z}_{k,j}}
%
%
%-------------------------------------------------------------------------------
%
%Punkt p_q, p_l
%
\newcommand{\pq}{\mathbf{p}_q} 
\newcommand{\pl}{\mathbf{p}_l}
%
%-------------------------------------------------------------------------------
%
%Measurement dimension:
%
\newcommand{\nz}{n_{\mathbf{z}}}
%
%------------------------------------------------------------------------------
%
%Beobachtung (Menge der Messungen) zum Zeitpunkt k
%Alle Beobachtungen (Gesamtheit der Mengen der Messungen) bis zum Zeitpunkt k
%
%ZB: \mathcal{Z}_{k}, \mathcal{Z}_{1:k-1}
\newcommand{\Zset}[1]{\mathcal{Z}_{#1}}
%
%-------------------------------------------------------------------------------
%
%Konstellation zum Zeitpunkt k:
%
\newcommand{\Const}[1]{\mathcal{T}_{#1}}
%
%-------------------------------------------------------------------------------
%
%probability density function of the measurement z at some time step (given as parameter)
%
\newcommand{\fApriZ}[1]{f_{\mathbf{Z}^{\mathbf{x}}_{#1}}}
\newcommand{\fApriZkxi}[2]{f_{\mathbf{Z}^{\mathbf{x}_{#2}}_{#1}}}
%
%-------------------------------------------------------------------------------
%
%Innovation: z Schlange:
%
% \innovation = \tilde{\mathbf{z}}
\newcommand{\innovation}{\tilde{\mathbf{z}}}
\newcommand{\Innovation}{\tilde{\mathbf{Z}}}
%
%-------------------------------------------------------------------------------
%
%Sch�tzungen f�r Zustand und Messung:
%
%
%Erwartungswert des Zustandes (entspricht der a-posteriori-Sch�tzung)
%
%Usage: \xExp{k} = \hat{\mathbf{x}}_{k}
\newcommand{\xExp}[1]{\hat{\mathbf{x}}_{#1}}
%
%
%a-priori sch�tzung des Zustandes zu einem bel. Zeitpunkt
%
\newcommand{\xApri}[1]{\hat{\mathbf{x}}_{#1}^-}
%
%
%a-posteriori sch�tzung des Zustandes zu einem bel. Zeitpunkt
\newcommand{\xApost}[1]{\hat{\mathbf{x}}_{#1}}
%
%
%Erwartungswert der Messung zu einem bel. Zeitpunkt
%
\newcommand{\zEst}[1]{\hat{\mathbf{z}}_{#1}}
%
%
%
%-------------------------------------------------------------------------------
%
%Assoziationen und Joint Events:
%
%\assoXZ = \theta^{\xii \rightarrow \zj}
\newcommand{\assoXZ}{\theta^{\xii \rightarrow \zj}}
%
%\asso{i}{j} = \theta^{\x_{i} \rightarrow \z_{j}}
\newcommand{\asso}[2]{\theta^{\x_{#1} \rightarrow \z_{#2}}}
%
%Joint Event:
\newcommand{\JE}{\Theta}
%
%
%
%-------------------------------------------------------------------------------
%
%Kovarianzen:
%
\newcommand{\Cov}{\textnormal{Cov}} 
%
%
%A-priori Zustandskovarianz zum bel. Zeitpunkt:
%
%Usage: PxApri{k} = \mathbf{P}_{\mathbf{x}_{k}^- \mathbf{x}^-_{k}}
\newcommand{\PxApri}[1]{\mathbf{P}_{\mathbf{X}_{#1}^- \mathbf{X}_{#1}^-}}
% Weitere m�gliche Schreibweisen: P^-_{k} = P_{x^-_k x^-_k} = P_{k | k-1}
%
%
%A-posteriori Zustandskovarianz zum Zeitpunkt {k}:
%
%Usage: PxApost{k}
\newcommand{\PxApost}[1]{\mathbf{P}_{\mathbf{X}_{#1} \mathbf{X}_{#1}}}
\newcommand{\PxApostC}[1]{\mathbf{P}^{c}_{\mathbf{X}_{#1} \mathbf{X}_{#1}}}
% Weitere m�gliche Schreibweisen: P^+_{k} = P_{x^+_k x^+_k} = P_{k | k}
%
%
%Zustandskovarianz
%
\newcommand{\Px}[1]{\mathbf{P}_{\mathbf{X}_{#1} \mathbf{X}_{#1}}}
%
%
%Innovations-Kovarianz:
%
\newcommand{\InnCov}[1]{\mathbf{P}_{\tilde{\mathbf{Z}}_{#1} \tilde{\mathbf{Z}}_{#1}}}
\newcommand{\Ptilde}[1]{\mathbf{\tilde{P}}_{#1}}%_{\mathbf{z}_{#1} \mathbf{z}_{#1}}}
%
%
%Sch�tzung der Mess-Kovarianz: (= Innovations-Kovarianz?)
\newcommand{\Pz}[1]{\mathbf{P}_{\mathbf{Z}_{#1} \mathbf{Z}_{#1}}}
%
%
%
%-------------------------------------------------------------------------------
%
%Probabilities:
%
%
%a-priori Probability with some index, (e.g. "FB" for "feature-based")
% P_{k|k-1}^{FB} oder P^{-}_{FB, k}
%
%Nutzung: \PaPri{FB}{k}
\newcommand{\PaPri}[2]{P_{#2|#2 - 1}^{#1}}
%\newcomand{\PaPri}[2]{P_{#1, #2}^{-}}

%a-posteriori Probability with some index, (e.g. "FB" for "feature-based")
% P_{k|k}^{FB} oder P_{FB, k}
%
%Nutzung: \PaPost{FB}{k}
\newcommand{\PaPost}[2]{P_{#2|#2}^{#1}}
%\newcomand{\PaPost}[2]{P_{#1, #2}}
%
%