\usepackage[english, ngerman]{babel}
\usepackage[latin1]{inputenc}
\usepackage{array}
\usepackage{multicol}
\usepackage[absolute,overlay]{textpos}
\usepackage[babel]{csquotes}

\usepackage{multimedia}

\usepackage[final]{pdfpages}	%final: Inserts pages. This is the default.
										%draft: Does not insert pages, but prints a box and the filename instead.
										%enable-survey: Activates survey functionalities.
										%               (experimental, subject to change)

\usepackage[open, openlevel=3]{bookmark}

\usepackage{tikz}

%Einbindung zu verwendender TikZ-Bibliotheken
%(in diesem Beispiel werden eigentlich nur die Bibliotheken
%shapes.misc und decorations.pathmorphing verwendet)
\usetikzlibrary{calc,arrows,positioning,shadows,backgrounds,chains,topaths,matrix,scopes,decorations,decorations.pathmorphing,decorations.markings,fadings,shapes.geometric,shapes.multipart,shapes.misc,through,automata,fit}

% Definition eines TikZ Objekt: sehr dicker, schwarzer Pfeil, mit Pfeilspitzentyp 'stealth'
\tikzstyle{lblock}=[ultra thick, -stealth, black]
